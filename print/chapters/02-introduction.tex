\begin{subchapter}{Single carrier transmision}
  A classic digital transmission standard will usually map
  digital bits to complex symbols and will modulate these
  symbols at a fixed rate onto a single carrier frequency. \\

  An example of six \acrshort{qam4} symbols that are transmitted
  in equal time intervals can be seen in figure \ref{img:qam4_symbol_time}.
  The Symbols are shown in the complex baseband domain and the
  hypothetical filter is constructed such, that the signal
  phase is interpolated linearily between symbols.

  \figurizefile{diagrams/qam4_symbol_time.tex}
               {img:qam4_symbol_time}
               {\acrshort{qam4} modulated symbols over time in the complex domain}
               {1}{H}

  Clock recovery and synchronizing to a continious stream of
  symbols are, at least in theory, straightforward.

  When only one of the complex components (real- or imaginary part)
  is observed, as shown in figure \ref{img:eye_diagram}, one can
  immidiately guess the optimal sampling time, indicated by the
  vertical line in the middle of the plot.
  The optimal sampling time can be found by observing the
  zero-crossings of the signal which have to happen half a
  sybol period after the sampling.

  In low noise situations, as shown in \ref{img:eye_diagram},
  the modulated digital signal can be perfectly reconstructed.

  \figurizegraphic{diagrams/eye_diagram.pdf}
                  {img:eye_diagram}
                  {Eye diagram}
                  {0.7}{H}

  A phenomenom that is quite common in terristrial communications
  is multipath propagation.
  What this means is that a signal that was transmitted once reaches the
  receiver multiple times delayed by different amounts of time.

  Figure \ref{img:multipath} shows an example of a scenario where
  multipath propagation is happening.
  The Transmitter does not have a direct line of sight to the receiver,
  so signals that reach the receiver did so by being reflected
  on objects.

  In the depicted scenario there are two paths by which signals from
  the transmitter can reach the receiver.
  The path in which the signal is reflected by the object on
  the top is longer than the other path, this means the signal will
  be delayed longer and damped more.

  \figurizefile{diagrams/multipath.tex}
               {img:multipath}
               {A Multipath scenario without a direct line of sight}
               {0.5}{H}

  Figure \ref{img:eye_diagram_multipath} shows the effect of the
  two signals superimposing on one another.
  The signal that follows the longer transmission path
  is delayed by $0.9$ symbol durations and has a magnitude
  of $\SI{10}{\percent}$, $\SI{30}{\percent}$, $\SI{50}{\percent}$ and
  $\SI{70}{\percent}$ of the original signal amplitude.

  % TODO: Phasendrehung und alles ist gar nicht beachtet
  \figurizefile{diagrams/eye_diagram_multipath.tex}
               {img:eye_diagram_multipath}
               {Eye diagrams}
               {0.9}{H}

  With increasing amplitude of the reflected signal the decoding
  and clock recovery become more and more difficult.
  To reduce the effects of multipath propagation one can
  increase the symbol durations, but this will in turn
  reduce the data rate, limiting the data throughput.
\end{subchapter}

\begin{subchapter}{Multicarrier transmision}
  As was already hinted at in the previous chapter a reduction
  in symbol rate does help mitigating the negative effects of
  multi-path propagation but does also reduce the possible datarate.

  Figure \ref{TODO} shows that halving the symbol rate also
  halves the width of the signal in the frequency domain,
  meaning that one could now fit two half-datarate modulated carriers in the
  same bandwidth previously occupied by one.

  As figure \ref{TODO} shows power from one carrier might leak
  into the other carrier, depending on the signal waveform and
  carrier spacing. Such \gls{isi} can be prevented by designing
  the carriers to be orthogonal.

  A straightforward scheme to generate multiple orthogonal
  carriers uses the \gls{ifft} to generate the time domain
  signals, as shown in figure \ref{TODO}.
  This scheme is also known as \acrlong{ofdm}.

  % TODO: a lot
\end{subchapter}

\begin{subchapter}{Cyclic prefix}
  So far we learned that \gls{ofdm} is modulation scheme
  that is useful to transfer high datarates over channels
  where multipath propagation would limit the throughput of
  a single-carrier scheme. \\

  There are however still some issues to consider when using
  \gls{ofdm} over a multipath channel. \\

  If one were to just sequentially output the \acrshort{ofdm}-symbols
  generated in the previous chapter one would encounter multipath-effects
  as shown in figure \ref{img:ofdm_multipath}.
  In the diagram the \acrshort{ofdm} symbols $-7$ to $0$ are received
  over two paths of different length.
  At the receiver the two signals will add up leading to interference
  between adjacent symbols.

  \figurizefile{diagrams/ofdm_multipath.tex}
               {img:ofdm_multipath}
               {\acrshort{ofdm} multipath without guard interval}
               {0.7}{H}

  To prevent interference between symbols a guard interval has
  to be introduced between symbols, as shown in figure \ref{img:ofdm_multipath}.
  The length of the guard interval has to be designed according to
  the targeted scenario. It should usually be at least as long as the
  impulse response of the targeted channel.
  Which in terms of multipath propagation is the time difference between
  the signal being received over the shortest path and over the longest path.

  \figurizefile{diagrams/ofdm_multipath_cp.tex}
               {img:ofdm_multipath}
               {\acrshort{ofdm} multipath with guard interval}
               {0.7}{H}

  Due to the cyclic nature of the \gls{fft} the guard interval
  is often filled with the last samples of the following \gls{ofdm}
  symbol. Guard intervals generated in this manner are called
  \gls{cycpf}.
\end{subchapter}

\begin{subchapter}{Schidl \& Cox Synchronization}
  Synchronization to a datastream in a single-carrier scheme
  could be performed as easily as observing zero-crossings in the
  timedomain to detect symbol transitions.
  In a multi-carrier system the raw samples are not available
  at the receiver prior to fourier-transformation to the
  frequency domain, thus synchronization has to be performed
  differently. \\

  One method to synchronize a receiver to a continuous stream
  of \gls{ofdm}-symbols is the introduction of null-symbols,
  \gls{ofdm}-symbols where the signal power is zero for the
  symbol duration.

  An example of such a symbol stream is shown in figure
  \ref{img:ofdm_zero_power_sync}.
  To synchronize to the symbol stream the receiver has
  to monitor the received input power for dips with a
  length of one symbol duration. \\

  This synchronization scheme is for example used in
  the \acrshort{dabplus} audio broadcasting standard.

  \figurizefile{diagrams/ofdm_zero_power_sync.tex}
               {img:ofdm_zero_power_sync}
               {\acrshort{ofdm} Synchronization using a null symbol}
               {0.7}{H}

  While being easy to implement this method does however
  have some drawbacks.
  For one it is not well suited for burst transfers,
  where the signal power of a received burst is not
  known a priori, complicating the selection of threshold
  values.
  Secondly it does not allow the receiver to estimate the
  frequency offset between its carrier and the carrier of the
  transmitter, a previously undiscussed topic, that is
  for example important when the receiver and transmitter
  are moving relative to one another, due to the resulting doppler
  shift in frequency. \\

  The gist of how a Schmidl \& Cox synchronization sequence
  is introduced into a stream of \gls{ofdm}-symbols is
  demonstrated in figure \ref{img:ofdm_sc_sync}.

  The Schmidl \& Cox sequence consists of a complex sequence
  of half the symbol length that is sent twice, back to back.

  \figurizefile{diagrams/ofdm_sc_sync.tex}
               {img:ofdm_sc_sync}
               {\acrshort{ofdm} Synchronization using a Schmidl \& Cox symbol}
               {0.7}{H}

  \figurizefile{diagrams/sc_detector_blocks.tex}
               {img:sc_detector_blocks}
               {Schmidl \& Cox detector block diagram}
               {0.7}{H}

  \figurizefile{diagrams/sc_detector_output_nooff.tex}
               {img:sc_detector_output_nooff}
               {TODO}
               {0.9}{H}

  \figurizefile{diagrams/sc_detector_output_fqoff.tex}
               {img:sc_detector_output_fqoff}
               {TODO}
               {0.9}{H}
\end{subchapter}
