This thesis demonstrates an implementation of the \acrshort{schcox}
algorithm optimized for processing speed, contained in
the XFDMSync\cite{xfdmsyncweb} library of GNU Radio modules
that was written as part of this thesis. \\

To demonstrate the special requirements imposed by multi-carrier
transmission methods, namely accurate synchronization in the
time and frequency domain and resilience against frequency-selective channcels,
a very basic introduction to multi-carrier systems will be given.
This introduction will cover the concepts of multipath propagation
in radio systems, \acrshort{ofdm} and cyclic prefixes. \\

To give an intuition on why advanced synchronization techniques
are needed in multi-carrier systems a simple input power based
synchronization method will be presented that does not fulfil the requirements
in frequency-domain synchronization and multipath-propagation resilience.
This synchronization method will then be compared to the \acrlong{schcox}
algorithm, that can, in addtion to providing synchronization in
the time domain, perform frequency-domain synchronizations and
and has superior multipath resilience. \\

Following the demonstration of the algorithms used there will be
an introduction of GNU Radio, a free and open source \gls{sdr}
processing framework.
This introduction will cover the different ways of using GNU Radio,
namely the \acrlong{grc}, the Python interface and the C++ interface,
and how GNU Radio can be used to create signal processing graphs
and nodes.
An appendix on this chapter will go into more detail on how
GNU Radio schedules its processing and how the buffer management
is optimized for processing speed. \\

Next the \acrlong{schcox} implementation written alongside
this thesis will be presented, demonstrating some considerations
made while translating the \acrshort{schcox} algorithm
into a fast software implementation. \\

Finally the performance of the \acrshort{schcox} implementation
will be evaluated using a set of experiments.
The experiments will test different aspects of the implementation.
Firstly the accuracy of the synchronization in time, in the presence
of different disturbances, namely white gaussian noise, a frequency offset
between transmitter and receiver and a frequency-selective channel.
Secondly the performance of the synchronization in the frequency-domain
will be tested in a hardware test over an actual channel.
And finally the processing performance of the implementation
in terms of maximum achievable sample rate on common computing
hardware. \\
