The main goal of the project accompanying this thesis was
to take the abstract formulation of the \acrlong{schcox}
synchronization algorithm and to transfer it into an actually
useable software product. \\

The result of this work is XFDMSync \cite{xfdmsyncweb},
a set of GNURadio processing blocks to be used for multi-carrier
synchronization tasks.

Writing this piece of software imposed some interesting
challenges not commonly seen in academic software development,
like concentrating on interoperability with an existing
software framework and spending a large proportion of
development time on optimizing for maximum data throughput. \\

As a result the blocks should be usable
by the general GNURadio userbase to synchronize onto
reasonably high-datarate data-streams. \\

Another aspect of this project was to quantify
the performance of the synchronization blocks at hand.
To perform these test a collection of testcases
were implemented to characterize the
time and frequency aspects of synchronization as
well as the computational throughput.

The components fared well in all of these tests
when compared to more naive synchronization
methods and provided sufficient throughput
even for demanding applications like realtime WiFi frame
detection. \\

Some aspects that were not explored in depth
and could be targets of future research work
were the optimization of detection parameters
like threshold levels for specific applications
or hardware tests on actual software-defined radio
devices.
