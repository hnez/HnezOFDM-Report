In order to give a better understanding of the special
requirements multi-carrier systems impose on the
synchronization between a transmitter and a receiver
this chapter presents the basic operating principles
of both methods in general and compares them with
a focus on synchronization. \\

This chapter also introduces the \acrlong{schcox}
synchronization algorithm that is later implemented
in software and some aspects of the preamble design.

\begin{subchapter}{Single carrier transmission}
  A classical single-carrier digital transmission standard
  usually maps digital bits to complex symbols from an
  alphabet $A$ and modulate these
  symbols at a fixed rate onto a carrier frequency. \\

  \autoref{img:qam4_symbol_time} presents an example of
  a signal where six \acrshort{qpsk} encoded symbols
  are transmitted in equal time intervals.
  In \acrshort{qpsk} the alphabet consists of the complex
  symbols

  \begin{equation*}
    A = \frac{1}{\sqrt{2}} \cdot \left\{ 1+j, 1-j, -1-j, -1+j \right\} \quad .
  \end{equation*}

  The symbols are shown in the complex baseband domain and a
  hypothetical filter is constructed such that the signal
  phase is interpolated linearily between symbols.

  \figurizefile{diagrams/qam4_symbol_time.tex}
               {img:qam4_symbol_time}
               {\acrshort{qpsk} modulated symbols over time in the complex domain}
               {1}{H}

  When only one of the complex components (real- or imaginary part)
  is observed and the continous signal is displayed as an eye-diagram,
  meaning it is cut and overlayed at symbol-length
  intervals, as shown in \autoref{img:eye_diagram}, one can
  immediately guess the optimal sampling time, indicated by the
  vertical line in the middle of the plot.
  The optimal sampling time can be found by observing the
  zero-crossings of the signal which have to happen half a
  symbol period before or after the sampling
  \footnote{This is a simplified formulation of the second Nyquist criterion}.
  In low noise situations, like in \autoref{img:eye_diagram},
  the modulated digital signal can be perfectly reconstructed
  when sampled at the correct points in time.

  \figurizegraphic{diagrams/eye_diagram.pdf}
                  {img:eye_diagram}
                  {Eye diagram of a low noise \acrshort{qpsk} signal}
                  {0.7}{H}
\end{subchapter}

\begin{subchapter}{Multipath propagation}
  A phenomenon that is quite common in terrestrial communications
  is multipath propagation.
  What this means is that a signal that was transmitted once reaches the
  receiver multiple times, delayed and attenuated by differing amounts. \\

  \autoref{img:multipath} shows a simplified multipath propagation example.
  In this example the transmitter does not have a direct line of
  sight to the receiver, due to an object in the middle of the scene,
  so signal rays may not propagate directly between transmitter and receiver.
  Instead signals that reach the receiver do
  so by reflecting off of other objects in the scene.

  In the depicted scenario there are two macro paths by which signals from
  the transmitter may reach the receiver, these macro paths consist of many
  micro paths.
  The effects of the macro paths are modeled as a combination of
  the micro paths it consists of.
  The macro path resulting from the reflection with the object on
  the top is longer than the other macro path, resulting in this signal being
  delayed and attenuated by a larger amount.

  \figurizefile{diagrams/multipath.tex}
               {img:multipath}
               {A Multipath scenario without a direct line of sight}
               {0.5}{h}

  Figure \ref{img:eye_diagram_multipath} shows the effect of the
  two delayed signals superimposing on one another, on the same signal
  as shown in \autoref{img:eye_diagram}.
  The signal arriving via the longer transmission path
  is delayed by $0.9$ symbol durations and has a magnitude
  of $\SI{10}{\percent}$, $\SI{30}{\percent}$, $\SI{50}{\percent}$ and
  $\SI{70}{\percent}$ of the original signal amplitude, respectively.

  \figurizefile{diagrams/eye_diagram_multipath.tex}
               {img:eye_diagram_multipath}
               {Eye diagrams}
               {0.9}{h}

  With increasing amplitude of the superimposed signal the optimal
  sampling points and zero crossings used for clock recovery
  become less and less obvious.
  A simple method to reduce the negative effects of multipath propagation is to
  increase the symbol durations, reducing the relative shift between
  the differently delayed signals, leading to a less
  distorted output.

  But doing so does obviously reduce the data rate,
  limiting the overall data throughput.
\end{subchapter}

\begin{subchapter}{Multi-carrier transmission}
  As was already hinted at in the previous sub-section a reduction
  in symbol rate helps mitigating the negative effects of
  multipath propagation but also leads to reduced datarate. \\

  A second positive effect of reducing the symbol rate,
  in addition to increasing multipath resilience is
  a reduction of the occupied bandwith.
  Halving the symbol rate leads to
  half the frequency range being occupied by the
  modulated signal.

  If one were to fill this newly freed bandwith with a
  second signal of equal bandwith without
  degrading the original signal one could
  transfer the same datarate while utilizing the
  same bandwith as in the single-carrier case
  modulated at twice the symbol rate,
  while still increasing the resilience to
  multipath propagation. \\

  In order to not degrade one another the two parallel
  sub-streams have to be modulated onto orthogonal
  carrier frequencies.
  One way of performing this modulation onto orthogonal
  frequencies is by using an \gls{ifft}, the inverse operation
  of the \acrlong{fft}.
  A chunk of $N$ symbols $s_{i}$ to be modulated onto respective carriers
  are used as an input vector to the \gls{ifft}.
  The \gls{ifft} then synthesizes the symbols onto $N$ orthogonal
  carrier frequencies as shown in \autoref{eq:ifftsynth} \cite{kammeyer2012}.

  \begin{equation}
    \label{eq:ifftsynth}
    x(k) = \sum_{i=0}^{N-1} s_{i} \cdot e ^ {2\pi j \cdot \frac{ik}{N}}
  \end{equation}

  The resulting time-domain signal can then be modulated onto
  a high frequency carrier to be sent by a transmitter.

  This scheme of synthesising symbols onto orthogonal
  frequencies, usually using an \gls{ifft}, and sequentially
  transferring the output chunks is commonly called \gls{ofdm}.
\end{subchapter}

\begin{subchapter}{Cyclic prefix}
  So far it was demonstrated that \gls{ofdm} is a modulation scheme
  that is useful to transfer high datarates over channels
  where multipath propagation would degrade the integrity of
  a single-carrier scheme. \\
  There are however still some issues to consider when using
  \gls{ofdm} over a multipath channel. \\

  If one were to just sequentially output the \acrshort{ofdm}-symbols
  generated in the previous chapter back-to-back one would encounter multipath-effects
  as shown in figure \ref{img:ofdm_multipath}.
  In the diagram the \acrshort{ofdm} symbols $-7$ to $0$ are received
  over two paths with different delays and equal attenuation.
  At the receiver the two signals will superimpose, leading
  to interference between adjacent symbols.

  \figurizefile{diagrams/ofdm_multipath.tex}
               {img:ofdm_multipath}
               {\acrshort{ofdm} multipath propagation without guard interval}
               {0.7}{H}

  To prevent interference between subsequent symbols a guard interval has
  to be introduced to separate them, as shown in figure \ref{img:ofdm_multipath}.
  The length of the guard interval has to be designed according to
  the targeted scenario, it should usually be at least as long as the
  impulse response of the targeted channel,
  which in terms of multipath propagation is the time difference between
  the signal being received over the shortest path and over the longest path\cite{kammeyer2012}.

  \figurizefile{diagrams/ofdm_multipath_cp.tex}
               {img:ofdm_multipath_cp}
               {\acrshort{ofdm} multipath with guard interval}
               {0.7}{H}

  Due to the cyclic nature of the \gls{fft}, used to separate the
  carriers at the receiver, the guard interval
  is often filled with the last samples of the following \gls{ofdm}
  symbol. Making the effects of the channel effectively cyclic
  as well.
  Guard intervals generated in this manner are called \gls{cycpf}.
\end{subchapter}

\begin{subchapter}{Synchronization using null-symbols}
  As presented earlier synchronization to a datastream in a single-carrier scheme
  may be performed as easily as observing zero-crossings in the
  timedomain to detect symbol transitions.
  In a multi-carrier system the actual sub-carriers are not available
  at the receiver prior to fourier-transformation,
  thus synchronization has to be performed
  differently. \\

  One method to synchronize a receiver to a continuous stream
  of \gls{ofdm}-symbols is the introduction a null-symbol,
  an \gls{ofdm}-symbols where the signal power is zero for the
  entire symbol duration.

  An example of such a symbol stream is shown in
  \autoref{img:ofdm_zero_power_sync}.
  To synchronize onto the symbol stream the receiver has
  to monitor the received input power for dips with a
  length of one symbol duration. \\

  This synchronization scheme is for example used in
  the \acrshort{dabplus} audio broadcasting standard \cite{dabstandard}.

  \figurizefile{diagrams/ofdm_zero_power_sync.tex}
               {img:ofdm_zero_power_sync}
               {\acrshort{ofdm} Synchronization using a null symbol}
               {0.7}{H}

  The start of a frame can then be detected using
  a method like the one shown in \autoref{eq:nulldetect},
  where the signal energy in a window before the detection point
  is subtracted from the signal energy in a window following the detection point.
  The output $d(k)$ has a maxmimum when the window before the detection
  point does not contain any input energy (null-symbol) and
  the window following the the detection points contains a
  non-null symbol.

  \begin{equation}
    \label{eq:nulldetect}
    d(k)= \sum_{n=0}^{N} \abs{x(k+n)}^2 - \sum_{n=-N}^{-1} \abs{x(k+n)}^2
  \end{equation}

  While being easy to implement this method does however
  have some drawbacks.
  For one it is not well suited for burst transfers,
  where the signal power of a received burst is not
  known a priori, complicating the selection of threshold
  values.
  Secondly it does not allow the receiver to estimate the
  frequency offset between its carrier and the carrier of the
  transmitter, a previously undiscussed topic, that is
  for example important when the receiver and transmitter
  are moving relative to one another, due to the resulting doppler
  shift in frequency.
\end{subchapter}

\begin{subchapter}{Schmidl and Cox}
  \label{chap:intr_scsync}

  A more advanced synchronization method is
  the \acrlong{schcox} algorithm that is presented
  in this sub-chapter.
  This algorithm was first presented in 1997 by
  Timothy M. Schmidl and Donald C. Cox \cite{schmidlcox}.

  The gist of how a Schmidl \& Cox synchronization sequence
  is introduced into a stream of \gls{ofdm}-symbols is
  shown in \autoref{img:ofdm_sc_sync}.
  In this example the Schmidl \& Cox sequence
  consists of a complex sequence
  of half a symbol length that is sent twice, back-to-back,
  as shown in \autoref{img:ofdm_sc_sync}.

  \figurizefile{diagrams/ofdm_sc_sync.tex}
               {img:ofdm_sc_sync}
               {Introduction of a \acrlong{schcox} preamble}
               {0.7}{H}

  The detection of the repeated sequence at the receiver works by delaying
  the input stream by half a symbol length, performing a complex conjuate
  multiplication with the original stream and averaging over a window with
  a length of half a symbol.
  A block diagram of this process is shown in \autoref{img:sc_detector_blocks}.

  \figurizefile{diagrams/sc_detector_blocks.tex}
               {img:sc_detector_blocks}
               {Schmidl \& Cox detector block diagram}
               {0.7}{H}

  Figure \ref{img:sc_detector_output_nooff} gives an intuition
  of what the output stream of this processing chain looks like
  over time when a synchronization sequence is received. \\

  The simulation consists of a \gls{schcox} sequence
  that is constructed from two halves of $512$ random samples
  and a $128$ samples long \acrlong{cycpf}.
  The signal is distorted by \gls{awgn}.
  The \gls{snr} is $\SI{0}{\deci\bel}$.

  The left plot shows the magnitude of the output of
  the moving average, while the right plot shows its phase.

  The absolute value follows a trapezoidal shape.
  Before and after the synchronization sequence is received
  the value is nearly zero, due to the result of the complex
  conjugate multiplication of the input and its delayed version
  being randomly distributed, leading to it being canceled out
  in the averaging step.

  \figurizefile{diagrams/sc_detector_output_nooff.tex}
               {img:sc_detector_output_nooff}
               {\acrshort{schcox} output $\abs{x}$ and $\arg{x}$ ($\Delta f=0$)}
               {0.9}{H}

  The conditions leading to the peak in the plot are best
  described in \autoref{eq:sc_peak}.
  Where $x$ is the received signal, consisting of the
  random noise component $n$ and the synchronization sequence $s$,
  and the sum represents the moving average in the block diagram,
  which is performed over half a symbol length $N$. \\

  As the synchronization sequence is repeated after $N/2$ samples
  $s_{i-N/2}$ is equal to $s_{i}$ and as $n$ is randomly distributed
  its effect will mostly be canceled out by the averaging step. \\

  \begin{align}
    \label{eq:sc_peak}
    \frac{1}{N/2} \sum_{i=0}^{N/2} x_{i} \cdot x_{i-N/2}^\ast &=
    \frac{1}{N/2} \sum_{i=0}^{N/2} \left(s_{i} + n_{i} \right) \cdot \left(s_{i-N/2} + n_{i-N/2} \right)^\ast \nonumber \\
    &= \frac{1}{N/2} \sum_{i=0}^{N/2} \left(s_{i} + n_{i} \right) \cdot \left(s_{i} + n_{i-N/2} \right)^\ast \nonumber \\
    &\approx \frac{1}{N/2} \sum_{i=0}^{N/2} s_{i} \cdot s_{i}^\ast
    = \frac{1}{N/2} \sum_{i=0}^{N/2} \abs{s_{i}}^2
  \end{align}

  The linear slopes leading to the peak are a result of the
  correlated symbols sliding into and out of the averaging
  window. \\

  If, due to doppler shift or a difference in the reference clocks,
  there is a offset frequency $\Delta f$ superimposed onto the
  synchronization sequence $s'$ where the sampling frequency is $f_s$

  \begin{equation*}
    s'_{i}= s_{i} \cdot e^{j\cdot 2 \pi \frac{\Delta f}{f_s}i}
  \end{equation*}

  and if the component of the input signal $x$ is ignored the
  block diagram in \autoref{img:sc_detector_blocks} can
  now be described by the new \autoref{eq:sc_peak_fqoff}.

  \begin{align}
    \label{eq:sc_peak_fqoff}
    \frac{1}{N/2} \sum_{i=0}^{N/2} s'_{i} \cdot {s'}_{i-N/2}^{\ast}
    &= \frac{1}{N/2} \sum_{i=0}^{N/2}
     \left( s_{i} \cdot e^{j\cdot 2 \pi \frac{\Delta f}{f_s}i} \right)
     \left( s_{i-N/s}^{\ast} \cdot e^{-j\cdot 2 \pi \frac{\Delta f}{f_s} \left( i- \frac{N}{2} \right)} \right) \nonumber \\
    &= \frac{1}{N/2} \sum_{i=0}^{N/2} s_{i} \cdot s_{i}^\ast \cdot e^{j\cdot 2 \pi \frac{\Delta f}{f_s} \frac{N}{2}} \nonumber \\
    &= \frac{1}{N/2} \sum_{i=0}^{N/2} \abs{s_{i}}^2 \cdot e^{j\cdot 2 \pi \frac{\Delta f}{f_s} \frac{N}{2}}
  \end{align}

  \autoref{eq:sc_peak_fqoff} shows that the frequency offset between transmitter
  and receiver can be estimated using the argument $\arg{x}$ of the complex output of the
  moving average block in \autoref{img:sc_detector_blocks}.

  \begin{equation*}
    \arg{x} = 2 \pi \frac{\Delta f}{f_s} \frac{N}{2} \implies \Delta f = \frac{f_s}{N \pi} \arg{x}
  \end{equation*}

  The magnitude and phase of the output values for a simulation with
  frequency offset can be seen in figure \ref{img:sc_detector_output_fqoff}.
  For the given frequency offset the magnitude is largely unaffected but
  the phase is now non-zero.

  \figurizefile{diagrams/sc_detector_output_fqoff.tex}
               {img:sc_detector_output_fqoff}
               {\acrshort{schcox} output $\abs{x}$ and $\arg{x}$ ($\Delta f \neq 0$)}
               {0.9}{H}
\end{subchapter}

Its ability to estimate frequency offsets in addtion to producing
exact timing synchronization between transmitters and
receivers makes the \acrshort{schcox} algorithm especially useful
in multi-carrier scenarios where carriers are spaced tightly and
frequency offsets lead to interference between carriers.

\begin{subchapter}{Cross-correlation based preamble detection}
  \label{sec:xcorrtheo}
  To further improve the accurcacy of the synchronization in time
  a cross-correlation based approach can be used, where the cross-correlation
  between the input samples and a stored reference-preamble is
  calculated.
  A flow diagram and an exemplary output of a cross-correlation
  applied to an input stream and a stored preamble are shown
  in the dotted segment in \autoref{img:xcorr_theory}. \\

  \figurizefile{diagrams/xcorr_theory.tex}
               {img:xcorr_theory}
               {Improved synchronization in time using cross- and auto-correlation}
               {1}{H}

  In addition to the central peak the cross-correlation output
  in \autoref{img:xcorr_theory} also shows two additional side-peaks.
  These side-peaks occur half a preamble length before and after
  the main peak and are a result of the repeating nature of
  the preamble. \\

  A method to suppress these side-peaks is to use the auto-correlation
  output of the \gls{schcox} detector and to multiply it with the
  output of the cross-correlation\cite{awoseyila}.

  This effectively masks out the side-peaks and leads to an output as
  shown in the rightmost sub-diagram in \autoref{img:xcorr_theory} containing
  a single sharp peak that can be used to accurately synchronize
  onto the preamble.
\end{subchapter}

\begin{subchapter}{CAZAC sequences}
  In order for the cross-correlation based synchronization technique,
  introduced in the previous subsection, to work optimally the
  autocorrelation of the preamble should be minimal for time-offsets $\Delta t$
  where $\abs{\Delta t} \neq \frac{N}{2}$. \\

  A class of sequences where this requirement is fulfilled are
  \gls{cazac} waveforms.
  One class of sequences in this \gls{cazac} class are
  Zadoff-Chu sequences\cite{zadoffchu},
  constructed using \autoref{eq:zchu}, where
  $n \in [0, N_\text{ZC})$, $u \in [0, N_\text{ZC})$,
  $N_\text{ZC}$ and $u$ are relatively prime, $q \in \mathbb{Z}$
  and $N_\text{ZC}$ is the length of the sequence\cite{zcwiki}.

  \begin{equation}
    \label{eq:zchu}
    x_\text{u}(n)= e^{- \frac{j \pi u n \left( n + 1 + 2q \right) }{N_\text{ZC}}}
  \end{equation}

  A property of \gls{cazac} sequences in general is that
  their autocorrelation is zero for $\Delta t \neq 0$.
  So if one constructs a \gls{schcox} preamble by repeating
  a \gls{cazac} sequence the autocorrelation is
  zero for $\abs{\Delta t} \neq \frac{N}{2}$.

  Another useful property of \gls{cazac} sequences is
  the their constant amplitude over time
  which is useful for example in the design
  of \gls{agc} hardware.
\end{subchapter}
