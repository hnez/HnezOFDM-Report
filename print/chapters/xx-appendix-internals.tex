
\begin{subchapter}{GNURadio Internals}
  As the code samples in listings \ref{lst:gnuradio_companion_minimal}
  and \ref{lst:square_cc} showed, GNURadio blocks do not have to
  perform any buffer management themselves to handle the streaming
  input and output ports. \\

  The buffer management and scheduling of when to execute the processing
  block is instead handled internally by GNURadio.
  As the processing blocks are only allowed to communicate with
  other blocks using the previously discussed input and output ports
  and a, not yet discussed, message passing interface they can also
  be executed concurrently on multiple CPU cores by the GNURadio scheduler
  without having to fear race conditions on shared data. \\

  \begin{subsubchapter}{Buffer management and Scheduling}
    \figurizegraphic{images/gnuradio_companion_bufferdemo.png}
                    {img:gnuradio_companion_bufferdemo}
                    {A minimal flowgraph to demonstrate buffer management}
                    {0.6}{H}

    For a flowgraph as shown in figure \ref{img:gnuradio_companion_bufferdemo}
    GNURadio will allocate two internal buffers, one for the ``Signal Source'' block
    to write into and the ``Multiply Const'' block to read from and one for the
    ``Multiply Const'' block to write into and the ``Null Sink'' block to
    read from \cite{grblogbuffers}. \\

    In the following diagrams the first buffer will be called ``Buffer A''
    and the second buffer will be called ``Buffer B''.
    The buffers in GNURadio are organized as circular buffers, this means
    that size of the occupied memory region remains constant during execution
    and that there is one pointer into the memory region where the next
    values should be written (\texttt{write\_ptr}) and one or more
    pointers pointing to the next value to be read (\texttt{read\_ptr}). \\

    If both pointers point to the same position the buffer is empty.
    If a pointer is incremented past the end of the memory region
    it wraps around to the beginning. \\

    Figure \ref{img:gr_round_buffers_a} shows the initial state of the
    two buffers in the current example, right when the execution of the
    flow graph starts.
    In both buffers \texttt{write\_ptr} and \texttt{read\_ptr}
    point to the same location, the beginning of the buffers,
    so both buffers are empty.

    \figurizefile{diagrams/gr_round_buffers_a.tex}
                 {img:gr_round_buffers_a}
                 {Initial state of the two buffers}
                 {0.7}{H}

    The GNURadio scheduler tries to keep all the buffers filled
    and will execute blocks until every buffer is full.
    Blocks that depend on input to process like the ``Multiply Const''
    or ``Null Sink'' block in the example cannot be executed
    unless there is data in their input buffers. \\

    To determine which block to execute next the scheduler
    asks every block, which does not have a completely filled
    output buffer, to estimate how many input items it needs to fill
    its output buffers. \\

    For synchronous blocks, like the ``Multiply Const'' block,
    this corresponds to a simple 1:1 mapping, as the block
    needs $n$ input values to produce $n$ output values. \\

    Blocks without output ports, like the ``Null Sink'' block,
    can be scheduled whenever there is data in their input
    buffers.
    Blocks without input ports, like ``Signal Source'' can
    always be scheduled, but might not actually produce the
    desired number of output values, for example when a
    hardware device did not produce enough samples. \\

    In the example the scheduler asks the ``Multiply Const''
    block how many input items it will need to fill the
    its output buffer of length $n_0$, the block answers with
    $n_0$ items.
    The scheduler can not provide that many input items,
    as the input buffer is empty.
    The scheduler will then sucessively halve the number
    of output items it requests $n_{i+1}=n_i/2$ util
    it determines that it can not fulfill the blocks
    input requirements. \\

    The scheduler then determines that the ``Signal Source''
    block can be executed, as it does not depend on any
    inputs.
    The ``Signal Source'' is executed and produces some
    output values that it puts into ``Buffer A'', it is assumed
    that the block was not able to fill the complete buffer.
    The states of the buffers after the ``Signal Source''
    block is executed is shown in figure \ref{img:gr_round_buffers_b}. \\

    The diagram shows some valid data in ``Buffer A'' that was
    not yet consumed, ``Buffer B'' remains empty.

    \figurizefile{diagrams/gr_round_buffers_b.tex}
                 {img:gr_round_buffers_b}
                 {Buffer states after ``Signal Source'' was executed}
                 {0.7}{H}

    In the next scheduling round ``Signal Source'' has data
    available in its input buffer to be processed.
    Figure \ref{img:gr_round_buffers_c} shows the buffer states
    after ``Signal Source'' was executed.
    ``Buffer A'' is completely drained and ``Signal Source'' has
    written some output items to ``Buffer B''.

    \figurizefile{diagrams/gr_round_buffers_c.tex}
                 {img:gr_round_buffers_c}
                 {Buffer states after ``Multiply Const'' was executed}
                 {0.7}{H}

    Once there is data in ``Buffer B'' the ``Null Sink'' block
    can be scheduled and it consumes all the available input samples.
    The state of the buffers is shown in figure \ref{img:gr_round_buffers_d}.

    \figurizefile{diagrams/gr_round_buffers_d.tex}
                 {img:gr_round_buffers_d}
                 {Buffer states after ``Null Sink'' was executed}
                 {0.7}{H}

    The scheduling decisions in this chapter are choosen to be
    as illustrative as possible. In an actual flowgraph there are
    further considerations to be made, like the possibility of scheduling
    multiple blocks concurrently on machines with multiple \gls{cpu} cores,
    buffers with multiple readers or multiple block producing data
    in the same scheduling cycle.
  \end{subsubchapter}

  \begin{subsubchapter}{Efficient circular buffers}
    As hinted at in listing \ref{lst:square_cc}, when a block
    is asked to fill or read a buffer it is passed a region in
    that buffer indicated by its starting address and size. \\

    An illustration is shown in figure \ref{img:gr_rb_mmap_a}.
    The block is asked to write \texttt{size} elements into
    the buffer, starting at \texttt{write\_ptr}.

    \figurizefile{diagrams/gr_rb_mmap_a.tex}
                 {img:gr_rb_mmap_a}
                 {Filling a memory region}
                 {0.7}{H}

    A problem arrises when the block should write or read
    at the end of the buffer.
    As the buffer is circular reads and writes beyond the
    end of the buffer should wrap to addresses at the start of the
    buffer.

    Figure \ref{img:gr_rb_mmap_b} shows a situation where
    a write to the passed region would lead to an overflow
    to memory not belonging to the buffer.

    \figurizefile{diagrams/gr_rb_mmap_b.tex}
                 {img:gr_rb_mmap_b}
                 {Overflowing the memory region}
                 {0.7}{H}

    A possible workaround is to execute the block
    twice whenever the buffer boundary is crossed,
    as shown in \ref{img:gr_rb_mmap_c}.

    This leads to worse performance, as the block might have
    to perform some initialization on every execution.
    Further problems arrise when the size of the buffer is not
    a multiple of the size of an item.

    \figurizefile{diagrams/gr_rb_mmap_c.tex}
                 {img:gr_rb_mmap_c}
                 {Scheduling twice to prevent overflowing}
                 {0.7}{H}

    GNURadio uses another approach, it instructs the operating system
    to map the buffer twice into the address space of the program,
    right after one another \cite{grrdbufmmap}.

    This leads to a memory layout like in figure \ref{img:gr_rb_mmap_d}.
    Reads and writes to addresses \texttt{0x1000} to \texttt{0x1fff} in the program
    will be redirect to the same physical memory as reads/writes to addresses
    \texttt{0x0000} to \texttt{0x0fff} by the computers \gls{mmu}.

    \figurizefile{diagrams/gr_rb_mmap_d.tex}
                 {img:gr_rb_mmap_d}
                 {Mapping the same memory region twice to prevent overflowing}
                 {0.7}{H}

    The scheduler can thus just pass the start address and the length of
    the data to write/read to the block without having to explicitly handle
    the wrapping.
  \end{subsubchapter}
\end{subchapter}
