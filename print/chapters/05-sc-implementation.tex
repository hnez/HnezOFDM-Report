The GNURadio blocks implemented as part of this thesis are
shown in \autoref{img:annotated_gnuradio_companion_detectors},
circled in red.

\figurizefile{diagrams/annotated_gnuradio_companion_detectors.tex}
             {img:annotated_gnuradio_companion_detectors}
             {The interface of the \gls{schcox} detectors}
             {0.95}{h}

The purpose of these blocks is to take a stream of input signals
and to add annotations to this stream whenever a synchronization-preamble
is detected.
\autoref{img:annotated_gnuradio_companion_detectors_plot} shows an example output
of the detector blocks, where an an annotation is placed right at the start of a
synchronization sequence.

\figurizefile{diagrams/annotated_grc_detectors_plot.tex}
             {img:annotated_gnuradio_companion_detectors_plot}
             {The output of the \gls{schcox} detectors}
             {0.95}{h}

As performance is a major design goal in the implementation of these blocks,
all of them are written in C++ and some optimizations are used to make them
perform well on a modern \acrshort{cpu}, like utilizing
\acrshort{simd} instructions.

\begin{subchapter}{Schmidl and Cox correlator}
  The functionality of the \acrlong{schcox} correlator
  block is shown in \autoref{img:sc_correlator_blocks}.
  The block takes one input signal, calculates the correlation
  as described in the \acrlong{schcox} chapter and normalizes it
  with the average input power in the analyzed duration. \\

  It also outputs a delayed version of the input sequence so that
  a peak at the correlation output corresponds to the start of a
  preamble and not its end.

  \figurizefile{diagrams/sc_correlator_blocks.tex}
               {img:sc_correlator_blocks}
               {Block diagram of the \acrlong{schcox} correlator}
               {0.6}{h}
\end{subchapter}

\begin{subchapter}{Schmidl and Cox tagger}
  The next step in the detection pipeline is to find peaks in
  the output of the \acrlong{schcox} correlator and to put
  a tag\footnote{A tag in GNURadio is a an annotation to a
  datastream that attaches metadata to a specific point in time}
  on them. \\

  Depending on the input noise-level the correlation output
  can be quite noisy. To prevent local fluctuations from being
  detected as multiple start-of-frame preambles and thus producing
  bursts of start-of-frame tags the \acrshort{schcox} tagger
  applies a user-defined amount of hysteresis to the detection
  process.

  \figurizefile{diagrams/sc_tagger_hysteresis.tex}
               {img:sc_tagger_hysteresis}
               {Detection levels of the \acrshort{schcox} tagger}
               {0.6}{h}

  An illustration is shown in \autoref{img:sc_tagger_hysteresis}.
  The upper and lower detection thresholds are provided by
  the user as a parameter to the block. \\

  Whenever the upper threshold is exceeded a detection window starts.
  The detection window ends when the input falls below the lower
  threshold.
  Exactly one tag will be placed in this detection window at
  the position of the highest magnitude.
  The sequence length parameter places a constaint onto the
  maximum length of this detection window which is necessary
  as the block needs to keep a copy of the samples inside the
  window in order to be able to put a tag anywhere inside of it.
\end{subchapter}

\begin{subchapter}{Cross-correlation tagger}
  The last detection step is used to increase the synchronization-in-time
  accuracy of the preamble detection and to prevent triggering on unknown
  preambles. \\

  It takes as inputs the sample stream that was tagged with
  start-of-frame tags by the \acrshort{schcox} tagger
  and a reference preamble.

  Whenever a start-of-frame tag is encountered in the input stream,
  a chunk of samples is extracted around the tag and the
  cross-correlation of this chunk and a preamble stored for reference
  is computed. \\

  If the maximum cross-correlation is below a preset threshold
  the tag that triggered the check is not forwarded to the
  next block.

  If the threshold is exceeded the tag is moved to the sample $\hat{k}_\text{new}$
  where the cross-correlation $c(k)$ is at maximum. \\

  \begin{equation*}
    \hat{k}_\text{new}= arg\,max\left( \, \abs{c(k)} \, \right)
  \end{equation*}

  \figurizefile{diagrams/xcorr_tagger_blocks.tex}
               {img:xcorr_tagger_blocks.tex}
               {In-depth operation of the cross-correlation tagger}
               {0.8}{h}

  A more detailed view of the inner workings of the cross-correlation
  tagger is shown in \autoref{img:xcorr_tagger_blocks.tex}. \\

  The computation of the cross-correlation is, as shown in the
  block-diagram, implemented using the \acrlong{fft} and its inverse.
  This method provides superior computation speed when dealing
  with large chunks of samples, when compared to the naive implementation
  of the cross-correlation \cite{kammeyer2012}. \\

  The extracted chunk of samples is zero-padded to a power-of-two size.
  This has two effects: for one the \acrshort{fft} algorithm performs
  best when operating on chunks that are a power-of-two in length.
  Secondly the zero padding effectively turns the circular cross-correlation
  calculation performed by the \acrshort{fft} into a normal cross-correlation
  \cite{kammeyer2012}. \\

  After the calculation of the \acrshort{ifft} the resulting cross-correlation
  contains two side lobes in addition the desired maximum.
  These side lobes are a result of the repeating nature of the preamble, as
  discussed in \autoref{sec:xcorrtheo}.

  To mask out these side lobes the cross-correlation is multiplied onto
  the the correlation output of the \acrlong{schcox} correlator,
  which has a triangular pattern with a peak around the desired
  maximum, as can be seen in \autoref{img:sc_tagger_hysteresis}.
\end{subchapter}
