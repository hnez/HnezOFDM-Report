The GNURadio blocks implemented as part of this thesis are
shown in figure \ref{img:annotated_gnuradio_companion_detectors},
circled in red.

\figurizefile{diagrams/annotated_gnuradio_companion_detectors.tex}
             {img:annotated_gnuradio_companion_detectors}
             {The interface of the \gls{schcox} detectors}
             {0.95}{h}

The purpose of these blocks is to take a stream of input signals
and to add annotations to this stream whenever a synchronization-preamble
is detected.
Figure \ref{img:annotated_gnuradio_companion_detectors_plot} shows an example output
of the detector blocks, where an an annotation is placed right at the start of a
synchronization sequence.

\figurizefile{diagrams/annotated_gnuradio_companion_detectors_plot.tex}
             {img:annotated_gnuradio_companion_detectors_plot}
             {The output of the \gls{schcox} detectors}
             {0.95}{h}

As performance is a major design goal in the implementation of these blocks,
all of them are written in C++ and some tricks are used to make them
perform well on a modern \acrshort{cpu}.

\begin{subchapter}{\acrlong{schcox} correlator}
  The functionality of the \acrlong{schcox} correlator
  block is shown in figure \ref{img:sc_correlator_blocks}.
  The block takes one input signal, calculates the correlation
  as described in the \acrlong{schcox} chapter and normalizes it
  with the average input power in the analyzed duration. \\

  It also outputs a delayed version of the input sequence so that
  a peak at the correlation output corresponds to the start of a
  preamble and not its end.

  \figurizefile{diagrams/sc_correlator_blocks.tex}
               {img:sc_correlator_blocks}
               {Block diagram of the \acrlong{schcox} correlator}
               {0.6}{h}


\end{subchapter}

\begin{subchapter}{\acrlong{schcox} tagger}
  The next step in the detection pipeline is to find peaks in
  the output of the \acrlong{schcox} correlator and to put
  a tag on them. \\

  Depending on the input noise-level the correlation output
  can be quite noisy. To prevent local fluctuations from being
  detected as multiple start-of-frame preambles and thus producing
  bursts of start-of-frame tags the \acrshort{schcox} tagger
  applies a user-defined amount of hysteresis to the detection
  process.

  \figurizefile{diagrams/sc_tagger_hysteresis.tex}
               {img:sc_tagger_hysteresis}
               {Detection levels of the \acrshort{schcox} tagger}
               {0.6}{H}

  An illustration is shown in figure \ref{img:sc_tagger_hysteresis}.
  The upper and lower detection thresholds are provided by
  the user. \\

  Whenever the upper threshold is exceeded a detection window is started.
  The detection window ends when the input falls below the lower
  threshold.
  Exactly one tag will be placed in this detection window at
  the position of the highest value. 
\end{subchapter}

\begin{subchapter}{Cross-correlation tagger}
  The last detection step is used to increase the time accuracy
  of the preamble detection and to prevent triggering on foreign
  preambles.

  \figurizefile{diagrams/xcorr_tagger_blocks.tex}
               {img:xcorr_tagger_blocks.tex}
               {TODO}
               {0.8}{H}

  
\end{subchapter}
